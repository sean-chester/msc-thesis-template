\documentclass{article}

\usepackage{fontawesome}
\usepackage{hyperref}

\usepackage{makecell}
\renewcommand{\cellalign}{cl}
% Table styling adapted from: https://tex.stackexchange.com/a/615088/84485
\usepackage[table]{xcolor}
\usepackage{tabularx}
\usepackage{enumitem}


\definecolor{blue}{HTML}{008ED7}
\definecolor{mygray}{gray}{0.75}
\definecolor{lightBlue}{HTML}{e5f7ff}

\usepackage[
  top=1cm,
  bottom=1cm,
  left=4cm,
  right=4cm,
  headheight=17pt, % as per the warning by fancyhdr
  includehead,includefoot,
  heightrounded, % to avoid spurious underfull messages
]{geometry}


\newcommand{\name}{A Student}
\newcommand{\email}{astudent@uvic.ca}


\usepackage{fancyhdr}
\pagestyle{fancy}
\fancyhf{} % clear all fields
\renewcommand{\headrulewidth}{1pt}
%\fancyhead[R]{\bfseries\sffamily\thepage}
\fancyfoot[C]{\bfseries\sffamily\thepage}
\fancyhead[L]{\nouppercase{\bfseries\sffamily{Breadth Statement}}}
\fancyhead[R]{\nouppercase{\bfseries\sffamily\name}}

\title{Breadth Statement}
\author{\name}
\date{%
    \faEnvelope\ \ \href{mailto:\email}{\email}\\%
    Algorithms for Mixed Parallelism in Software ({\faBolt}AMPS) Lab\\%
    Department of Computer Science\\%
    University of Victoria, Canada\\[2em]%
    \today
}


\begin{document}

\maketitle

\begin{table}[h!]
  \vspace{-2em}
  \centering
    \begin{tabular}{rl}
      {\bfseries Supervisor} & Dr. Sean Chester\\
      {\bfseries Committee} & TBD\\
      {\bfseries Program} & Industrial Master's\\
      {\bfseries Program Start Date} & September 2023
  \end{tabular}
\end{table}

\section{Primary Research Area}

\noindent
{\bfseries Breadth Area}: Systems

\medskip\noindent
{\em Justification}\hspace{0.5em}
Five or so sentences about your research topic and why the research methods fit with the area of Systems (or another selection).
Five or so sentences about your research topic and why the research methods fit with the area of Systems (or another selection).
Five or so sentences about your research topic and why the research methods fit with the area of Systems (or another selection).
Five or so sentences about your research topic and why the research methods fit with the area of Systems (or another selection).
Five or so sentences about your research topic and why the research methods fit with the area of Systems (or another selection).




\begin{table}[htb]
\setlist{nolistsep}
\begin{tabularx}{\textwidth}[t]{XX}
\arrayrulecolor{mygray}\hline
\arrayrulecolor{blue}\hline
\rowcolor{lightBlue} \textbf{\textcolor{blue}{Applications}} & \\
\arrayrulecolor{mygray}\hline
\arrayrulecolor{blue}\hline

\makecell{%
~\\
\parbox{\linewidth}{%
A.1: CSC 501
}\\
~\\
}& 
\makecell{%
~\\
\parbox{\linewidth}{%
Algorithms \& Data Models\\
Fall 2023 (Dr. Sean Chester)\\
A+
}\\
~\\
}\\


\arrayrulecolor{mygray}\hline
\arrayrulecolor{blue}\hline
\rowcolor{lightBlue} \textbf{\textcolor{blue}{Systems}} & \\
\arrayrulecolor{mygray}\hline
\arrayrulecolor{blue}\hline

\makecell{%
~\\
\parbox{\linewidth}{%
S.1: CSC 591
}\\
~\\
}& 
\makecell{%
~\\
\parbox{\linewidth}{%
Directed Studies\\
Lock-free, GPU, and vector programming\\
Fall 2023 (Dr. Sean Chester)\\
A+
}\\
~\\
}\\

\arrayrulecolor{mygray}\hline

\makecell{%
~\\
\parbox{\linewidth}{%
S.2: CSC 586C\\
{\em (planned)}
}\\
~\\
}& 
\makecell{%
~\\
\parbox{\linewidth}{%
Topics Comp. Sys. Softw.\\
GPU Computing\\
Fall 2024 (Dr. Sean Chester)\\
N/A
}\\
~\\
}\\


\arrayrulecolor{mygray}\hline
\arrayrulecolor{blue}\hline
\rowcolor{lightBlue} \multicolumn{2}{l}{%
\textbf{\textcolor{blue}{Theory}}} \\
\arrayrulecolor{mygray}\hline
\arrayrulecolor{blue}\hline

\makecell{%
~\\
\parbox{\linewidth}{%
T.1: CSC 586B\\
{\em (planned)}
}\\
~\\
}& 
\makecell{%
~\\
\parbox{\linewidth}{%
Topics Comp. Sys. Softw.\\
Geometric Modelling\\
Fall 2024 (Dr. Teseo Schneider)\\
N/A
}\\
~\\
}\\

\arrayrulecolor{mygray}\hline
\arrayrulecolor{blue}\hline

\end{tabularx}
\caption{\label{table:courses-taken}List of taken and proposed coursework, ordered by breadth category}
\end{table}



\section{Coursework}
\autoref{table:courses-taken} lists both planned an completed coursework, organised by breadth category. The following subsections describe each course and its relationship to my primary research.

\subsection{Course A.1: CSC 501}
\noindent
{\em Course Description}\hspace{0.5em}
Three or so sentences taken from the course outline describing what this course is.
Three or so sentences taken from the course outline describing what this course is.
Three or so sentences taken from the course outline describing what this course is.

\medskip
\noindent
{\em Course Evaluation}\hspace{0.5em}
Four or so sentences describing how the course was evaluated and why that fits into the selected breadth category.
Four or so sentences describing how the course was evaluated and why that fits into the selected breadth category.
Four or so sentences describing how the course was evaluated and why that fits into the selected breadth category.
Four or so sentences describing how the course was evaluated and why that fits into the selected breadth category.


\medskip
\noindent
{\em Research Overlap}\hspace{0.5em}
Four or so sentences describing why this course does not overlap with your credit for your research project/thesis.
Four or so sentences describing why this course does not overlap with your credit for your research project/thesis.
Four or so sentences describing why this course does not overlap with your credit for your research project/thesis.
Four or so sentences describing why this course does not overlap with your credit for your research project/thesis.

\subsection{Course A.1: CSC 501}
As before.

\subsection{Course A.1: CSC 501}
As before.

\subsection{Course A.1: CSC 501}
As before.

\section{Supervisor's Statement}
\vspace{15em}


\begin{thebibliography}{1}

\bibitem{aa} Department of Computer Science Graduate Committee (n.d.). ``Breadth Requirement Classification.'' Retrieved 2-Jul-2024 from: \url{https://www.uvic.ca/ecs/computerscience/assets/docs/grad/breadth-requirement-classification.pdf}.

\end{thebibliography}

\end{document}
